\chapter{Resultados}
% Label para referenciar
\label{resultados}

\section{Análise explorátoria de dados} Na etapa de \ac{EDA}, utilizando os dados que foram tratados e armazenados nas etapas anteriores, será possível iniciar as análises estatísticas com o intuito de entender as distribuições, padrões e correlações existentes na massa de dados do \textit{Data Lake}. É uma etapa fundamental do processo, uma vez que é nessa etapa que o conhecimento é efetivamente gerado. Temos o objetivo de possibilitar a análise da atividade política de cada parlamentar de Câmara dos Deputados Federal a partir de dados oficiais, possibilitando uma visão clara e objetiva de gastos incomuns, quais os termos mais utilizados, 

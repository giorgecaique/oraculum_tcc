%\chapter{Desenvolvimento}
% Label para referenciar
%\label{desenvolvimento}

% Texto do capítulo

%Esta seção apresenta o desenvolvimento deste trabalho. Cada etapa da metodologia sera detalhada e apresentada atraves de seu codigo e observação

%\section{Classe utilitaria}

%Durante o desenvolvimento houve a necessidade de se criar um controle de versão dos arquivos para corrigir um problema de codificação do texto coletado do twitter devido a acentuação. Para controlar os arquivos de texto originais e traduzidos de cada candidato, foi criado uma classe utilitária denominada UTIL com dois atributos como podemos ver respectivamente nas linhas 2 e 3 , um foi utilizado para armazenar o nome do usuário do twitter e a outro em qual a versão dos arquivos se encontra. Assim esses atributos podem ser acessados de forma global por todas as outras classes do projeto, permitindo quando precisar criar um arquivo  daquele candidato.


%\section{Coleta de dados}

%O primeiro passo para o desenvolvimento do sistema foi a implementação da captura dos tweets do usuário que será submetido à análise. A captura foi realizada por meio da API disponibilizada pelo Twitter, utilizando o método \emph{GET statuses/user/timeline}, o qual retorna os 200 últimos \emph{tweets}. Esse valor é o maximo permitido por requicoes. Desta forma foi nessesario fazer uma nova requisiçao ja que 200 twitter nao é suficiente para uma analise confiavel.Podemos perceber que foi realizada uma requisição à API passando como parâmetro o último \emph{tweet} obtido pela primeira requisição. A segunda requisição coleta os próximos 200 \emph{tweets}. Apenas o texto com o conteúdo dos twittes do json retornado foi salvo no arquivo.

%O primeiro passo para o desenvolvimento do sistema foi a implementação da captura dos tweets do usuário que será submetido à análise. A captura foi realizada por meio da API disponibilizada pelo Twitter, utilizando o método \emph{GET statuses/user/timeline}, o qual retorna apenas os últimos  200 \emph{tweets}, esse e o valor máximo por cada requisição. Como esse valor e insuficiente para uma análise confiável foi feita uma nova chama ao método para coletar mais \emph{tweets} do usuário. Porem para isso é preciso passar como parâmetro na URL qual foi o último \emph{twitter} coletado para que a nova requisição traga apenas os \emph{twittes} que não foram coletados. Veja no Código 2.
%\begin{center}
%\begin{lstlisting}[title=Codigo 2 - Classe TwitterApi]
%class TwitterAPI:
%  def getTwitts(self):
%    session = OAuth1Session(self.CONSUMER_KEY, self.CONSUMER_SECRET, self.ACCESS_TOKEN_SECRET, self.ACCOUNT_ID)    
%    response = session.get('https://api.twitter.com/1.1/statuses/user_timeline.json?screen_name=' + Util.usuario + '&count=200')
%    tweets = json.loads(response.content)   
%    ultimo = tweets[len(tweets) -1]['id_str'].encode('utf-8')
%    arq = open(Util.usuario + Util.versao + '_PT-BR.txt', 'w')
%    linha = 0
%    while linha < len(tweets):
%      arq.write(tweets[linha]['text'].encode('utf-8') + "\n")
%      linha += 1
%    response = session.get('https://api.twitter.com/1.1/statuses/user_timeline.json?screen_name=' + Util.usuario + '&count=200&max_id=' + ultimo)
%    tweets = json.loads(response.content)
%   linha = 0
%    while linha < len(tweets):
%      arq.write(tweets[linha]['text'].encode('utf-8') + "\n")
%      linha += 1
%    arq.close()       
%\end{lstlisting}
%\textbf{\footnotesize Fonte: Criada pelo autor}
%\end{center}


%\section{Tradução do Texto}

%A segunda parte do desenvolvimento do código foi a implementação da tradução dos \emph{tweets} obtidos utilizando a API do Google Tradutor. Primeiramente o algoritimo lê o arquivo em Português e em seguida o envia para a API e salva o texto traduzido em um novo arquivo.

%A segunda parte do desenvolvimento do código foi a implementação da tradução dos \emph{tweets} . Foram testadas algumas Api gratuita porem todas tinha um limite caracteres ou por requisição ou cota diária o que todos eram insuficientes para este trabalho então foi utilizado a API  paga do Google Tradutor.

%Primeiramente o algoritmo lê o arquivo em Português daquele usuário e versão definida na classe UTIL. Em seguida o envia para a API através de uma requisição post onde é passado as chaves, qual é o idioma do texto original e do texto a ser traduzido como mostra a linha 2 do código 3. Após a resposta, o texto traduzido e salvo em um novo arquivo utilizando os padrões definidos.

%\begin{center}

%Antes de salvar o texto traduzido no arquivo, foi necessário converter sua codificação para UTF-8 para garantir que não haja problema com acentuação

%\section{Preparacao dos dados}

%Não foi preciso implementar, porque devido aos testes realizados durante o desenvolvimento, percebemos que a própria API do \emph{Personality Insight} não analisa os dados que ela não comprende.

%A terceira etapa do desenvolvimento definido na metodologia não foi precisa ser implementada, porque devido aos testes realizados durante o desenvolvimento e pesquisas feitas sobre o algoritmo utilizado no \emph{Personality Insight} as palavras que não se encontra no dicionário da língua inglesa não são consideradas na avaliação do perfil. 

%\section{API Personality Insight}

%A Classe Personality Insight apresenta o método que realiza a requição ao \emph{Bluemix} para a análise de personalidade. A API exige que seja enviado um \emph{profile} contendo o texto a ser analisado, qual idioma deseja obter os resultados, entre outras informaçoes opcionais. A API retorna um JSON com todos os resultados, que serão salvos em um outro arquivo de texto.

%No último método novamente lemos um arquivo porem dessa vez o que possui o texto em inglês e em seguida é feito a requisição a ao Personality Insight. A api da ibm  e precisa ser configurada com os parâmetro da versão a ser utilizada no caso foi a v3, a última já desenvolvida, além do usuário e senha do desenvolvedor no  Bluemix . Já a requisição e feita através do método profile que recebe o texto em inglês a ser analisado, o formato dele, quais serão os resultados retornados, e o idioma de saída no caso português do Brasil. Após a requisição a resposta e salva no formato json original com todos os dados obtidos em um outro arquivo com o mesmo nome de usuário e versão dos arquivos de texto português e inglês.


%\section{Correlação dos resultados dos testes}

%\begin{table}[H]
%\centering
%\caption{My Personality 100-item x Personality Insight }
%\label{Correlação Candidato 1}
%\resizebox{\textwidth}{!}{%
%\begin{tabular}{@{}cccc@{}}
%\toprule
%Candidato & Correlação   \\ \midrule
%Candidato 1 & 0,42 &  \\
%Candidato 2  & 0,28 \\
%Candidato 3  & 0,36 \\
 %\bottomrule
%\end{tabular}%
%}
%\end{table}

\chapter{Introdução}
% Label para referenciar
\label{introducao}
% Texto do capítulo

\section{Problema}
A verdade pode ser definida de duas formas. A verdade racional, definida a partir de um embasamento científico, e a verdade factual, esta, por sua vez, se apoiando nas relações humanas. Esta segunda possui maior importância no âmbito político, uma vez que a verdade racional é, a não ser que se prove através dos mesmos critérios aplicadas pelo método científico, irrefutável. Logo, ao prover a ambiguidade da verdade, se torna um importante artifício para a articulação de atos políticos \cite{entrepassadofuturo}.

É sabido que com os avanços da tecnologia, os regimes democráticos ao redor do globo vêm sofrendo com a desinformação da população. Observa-se, por exemplo, a influência que as \textit{fake news} e os \textit{bots} possuem nos processos eleitorais das sociedades democráticas contemporâneas \cite{fakenewsbot}.

Ao final da campanha presidencial americana de 2016, podê-se observar que as 20 notícias falsas com maior engajamento na internet possuíam mais compartilhamentos, \textit{likes} e comentários do que as 20 reportagens produzidas pela mídia tradicional com melhor desempenho no Facebook \cite{facebookamericananalysis}. 

Em seus primeiros doze meses de governo, o atual presidente da república do Brasil, Jair Messias Bolsonaro, deferiu 608 declarações falsas ou distorcidas em discursos, entrevistas e mídias sociais. Do total de declarações do presidente, 56\% delas possuem algum grau de distorção \cite{declaracoesbolsonaro}.

Em contrapartida, observa-se também uma discussão sobre como a tecnologia pode prestar um papel crucial nos processos decisórios em uma Democracia Digital. Como novos modelos de tomadas de decisão do Estado podem ser adaptados com o intuito de promover uma maior participação e engajamento da esfera civil \cite{democraciadigital}.

Um cenário onde a população está participando ativamente das atividades políticas, mas que também sofre com as estratégias que promovem a desinformação como artifício político. Considerando essa nova realidade, presume-se que informar a população através de fontes oficiais torna-se uma alternativa coerente.

\section{Objetivo Geral}
O objetivo deste trabalho consiste na criação de um repositório unificado contendo informações tratadas, verificadas e referenciadas que contemplem uma visão geral das atividades exercidas no Legislativo brasileiro, especificamente, a Câmara dos Deputados. A partir deste repositório, foram realizadas análises exploratórias que descrevem informações relacionadas às atividades parlamentares, especificamente analisando os gastos reembolsando as despesas dos deputados, as proposições submetidas à Câmara, temas dessas proposições, conteúdo dos \textit{tweets} dos perfis oficiais dos parlamentares além das notícias que os citam. Análises mais criteriosas relacionando o conteúdo das proposições com as notícias dos deputados e suas atividades no Twitter foram aplicadas com uma amostra dos deputados.

\section{Objetivos Específicos}
Visando atingir o objetivo geral, os objetivos específicos são:
\begin{enumerate} 
 \item [a)] Mapear as fontes de dados que serão utilizadas; 
 \item [b)] Elencar as tecnologias necessárias para o desenvolvimento;
 \item [c)] Projetar a arquitetura deste projeto;
 \item [d)] Implementar o projeto;
 \item [e)] Gerar indicadores gerais das atividades dos parlamentares;
 \item [f)] Elencar deputados para análises mais criteriosas.
\end{enumerate} 

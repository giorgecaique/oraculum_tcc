\chapter{Introdução}
% Label para referenciar
\label{introducao}
% Texto do capítulo

\section{Problema}
A verdade pode ser definida de duas formas. A verdade racional, definida a partir de um embasamento científico, e a verdade factual, essa, por sua vez, se apoiando nas relações humanas. Essa segunda possui maior importância no âmbito político, uma vez que a verdade racional é, a não ser que se prove através dos mesmos critérios aplicadas pelo método científico, irrefutável. Logo, ao prover a ambiguidade da verdade, se torna um importante artifício para a articulação de atos políticos \cite{entrepassadofuturo}.

É sabido que mesmo com os avanços da tecnologia, os regimes democráticos ao redor do globo vêm sofrendo com a desinformação da população, tornando-se um fator de grande influência durante os processos eleitorais.

Ao final da campanha presidencial americana de 2016, podê-se observar que as 20 notícias falsas com maior engajamento na internet possuíam mais compartilhamentos, likes e comentários do que as 20 reportagens produzidas pela mídia tradicional com melhor desempenho no Facebook \cite{facebookamericananalysis}. 

Em seus primeiros doze meses de governo, Bolsonaro deferiu 608 declarações falsas ou distorcidas em discursos, entrevistas e mídias sociais. Do total de declarações do presidente, 56 por cento delas possuem algum grau de distorção \cite{declaracoesbolsonaro}.

Considerando a permanência da desinformação como um fator político nas sociedades democráticas, fator que foi aperfeiçoado pelos avanços nas tecnologias de Análise de Dados e pelas mudanças nos paradigmas de mercado, presume-se que informar a população através de fontes oficiais torna-se uma alternativa coerente.

\section{Objetivo Geral}
O objetivo deste trabalho consiste na criação de um repositório unificado contendo informações tratadas, verificadas e referenciadas que contemplem uma visão geral das atividades exercidas no Legislativo brasileiro, especificamente, a Câmara dos Deputados Federal. Para a criação de um repositório onde um volume massivo de dados será utilizado para alimentar este ambiente e o acesso às informações diversas deve ser facilitado, será necessária a utilização de tecnologias e conceitos emergentes no cenário tecnológico.

\section{Objetivos Específicos}
Visando atingir o objetivo geral, os objetivos específicos são:
\begin{enumerate} 
 \item [a)] Mapear as fontes de dados que serão utilizadas; 
 \item [b)] Projetar a arquitetura desta solução;
 \item [c)] Levantar as tecnologias necessárias para o desenvolvimento;
 \item [d)] Implementar a solução;
 \item [e)] Levantar indicadores gerais das atividades dos parlamentares;
 \item [f)] Elencar deputados para análises mais criteriosas.
\end{enumerate} 

% Resumo em portugues
\setlength{\absparsep}{18pt} % ajusta o espaçamento dos parágrafos do resumo
\begin{resumo}
        Este estudo irá abordar a modelagem e implementação de uma arquitetura de Data Lake com o intuito de possibilitar a análise de dados referentes à Câmara dos Deputados do Brasil, à qual faz parte do Poder Legislativo da União. Utilizaremos técnicas de Big Data e Cloud Computing com o intuito de propor o estado da arte para uma arquitetura para análise de dados políticos. Serão utilizadas como fontes de dados os dados abertos disponibilizados pela Câmara dos Deputados, o monitoramento das atividades dos parlamentares nas redes sociais, e as notícias que referenciem os políticos do Legislativo. Iremos definir ao longo do artigo as tecnologias utilizadas e o seus benefícios para um projeto de Big Data.
        
    
    \textbf{Palavras-chave}: Big Data, Câmara dos Deputados, Twitter, Data Lake, Cloud Computing.
\end{resumo}


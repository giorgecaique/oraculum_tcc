% Resumo em portugues
\setlength{\absparsep}{18pt} % ajusta o espaçamento dos parágrafos do resumo
\begin{resumo}
        Este estudo aborda a modelagem e implementação de uma plataforma que possibilite a análise de dados referentes a Câmara dos Deputados do Brasil, a qual faz parte do Poder Legislativo da União. Utilizou-se de técnicas de Big Data e Cloud Computing com o intuito de propor o estado da arte para uma arquitetura de análise de dados políticos. Foram utilizadas como fontes de dados os dados abertos disponibilizados pela Câmara dos Deputados, o monitoramento das atividades dos parlamentares nas redes sociais, e as notícias que referenciem os políticos do Legislativo. A partir desta base, análises descritivas sobre as atividades parlamentares foram realizadas, além de estudos que identificam as correlações entre os discursos políticos dos deputados na mídia e o que é efetivamente realizado na Câmara.
        
    
    \textbf{Palavras-chave}: Big Data, Câmara dos Deputados, Twitter, Data Lake, Cloud Computing.
\end{resumo}

